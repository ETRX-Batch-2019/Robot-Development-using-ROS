\documentclass[11pt, a4paper]{beamer}
%\usetheme{Berkeley}
%\usetheme{Berlin}
%\usetheme{Copenhagen}
%\usetheme{Antibes}
%\usetheme{Darmstadt} 
\usetheme{Dresden} 

\usecolortheme{sidebartab}
\usepackage[]{movie15}
\usepackage{hyperref}
\hypersetup
{
	colorlinks=true,
	linkcolor=blue,
	filecolor=magenta,
	urlcolor=cyan,
}
\urlstyle{same}

\usepackage{graphicx}
\graphicspath{{./images/}}

\usepackage{movie15}

\begin{document}
	\setbeamertemplate{sidebar left}{}
	\title{Robot development using ROS}
	%\subtitle{ \\ Robot development using ROS}
	
		\author
		{Chirag Shah\\ Srijal Poojari\\
		 \textbf{Guide:} Kumar Khandale			
		}

	\institute{Sardar Patel Institute of Technology}
	
	\date{\today}
	%\addtobeamertemplate{sidebar left}{}{\includegraphics[scale = 0.3]{logowithtext.png}}
	\frame{\titlepage}

\setbeamertemplate{sidebar left}[sidebar theme]
\section{Overview of Project}
\begin{frame}{Overview of Project}

	
	\noindent This project deals with the exploring the ROS framework for development of a robotic system with various sensors and actuators in order to understand the underlying concepts and to create a robot/quadcoptor capable of forming a 3D map of a given environment using a depth camera (Microsoft Kinect).
	\newline
	
\end{frame}

\section{What is ROS}
\begin{frame}{What is ROS? }

The Robot Operating System (ROS) is a flexible framework for developing software  with tools, libraries and conventions that facilitate the creation of complex robot behaviour on a wide variety of robotic platforms.	
	
\end{frame}

\section{Weekly Plan}
\begin{frame}{Weekly Plan }

		\begin{tabular}{|c|c|c|}
			\hline
			\textbf{Week} & \textbf{Objective}\\
			\hline
			1 to 3 & Installing ROS, Understanding the underlying concepts. \\
			\hline
			4 to 5 & Interfacing the Microsoft Kinect sensor on Gazebo simulator.\\
			\hline
			6 to 8 & Understanding Pointclouds and RTAB Map for real time \\ & environment mapping. \\
			\hline
			9  &  Interface the Kinect sensor for mapping of \\ & real world environments.\\
			\hline
			
			
			
			

		\end{tabular}	
\end{frame}

\section{Thank You}
	\begin{frame}{Thank You}
	\centering \textbf{THANK YOU !}
	\end{frame}


\end{document}
