\documentclass{article}
\pagenumbering{arabic} 
\usepackage{graphicx}
\usepackage[a4paper, portrait, total={6.5in, 9in}]{geometry}
\graphicspath{{./images/}}
\newcommand\tab[1][1cm]{\hspace*{#1}}

\usepackage{hyperref}
\hypersetup
{
	colorlinks=true,
	linkcolor=blue,
	filecolor=magenta,      
	urlcolor=cyan,
}
\urlstyle{same}

\usepackage{enumitem}
\setlist{nosep}

\begin{document}
	\begin{center}
		\huge\textbf{Robot development using ROS}
	\end{center}
	
	\hrulefill
	
	\section*{Project Description:}
		The Robot Operating System (ROS) is a flexible framework for developing software  with tools, libraries and conventions that facilitate the creation of complex robot behaviour on a wide variety of robotic platforms.	
		\newline
		
		\noindent This project deals with the exploring the ROS framework for development of a robotic system with various sensors and actuators in order to understand the underlying concepts and to create a robot/quadcoptor capable of forming a 3D map of a given environment using a depth camera (Microsoft Kinect).
		\newline
			
	\section*{Work Plan:}
	\begin{itemize}
		\item Explore the ROS platform
		\item Interface a depth camera with ROS
		\item Create a 3-dimensional map of an indoor environment in simulation using a depth camera (like Microsoft Kinect), mounted on a robot/quadcoptor
		\item Implement the same for an actual real-world scenario
		
		\item Interface actuators and other sensors 
		\item Design and build a mobile robot/quadcopter with the above functionality controlled with ROS 
	\end{itemize}

	\section*{Challenges:}
	\begin{itemize}
		\item Learning ROS
		\item Implementing the SLAM algorithm
		\item Designing and building a robot/quadcopter
	\end{itemize}
	
\end{document}