\chapter{Conclusions}
An load emulation environment has been presented which is capable of simulating a electrical load in real-time. In this research the fundamental objective of the load emulation is to
provide a simulated electrical load to allow an inverter
to be tested at real power levels without the requirement of an actual load. 
The load emulation replaces the actual load during the testing and
development stages of the inverter design, thus providing a safer and more flexible development environment.
The ability to simulate an electrical load in real
time is one of the key elements which facilitates in the load emulation. \par
The primary analysis and results show that acceptable accuracy can be achieved in real time using a digital signal processor dedicated to this task.
The second requirement of the load emulation is the
ability to draw current from the inverter equal to that
predicted by the real time load model. To do this, the
load emulation incorporates its own internal bidirectional converter which acts as a controllable voltage
source. A current control loop ensures that this converter together with three-phase line inductors draw the
appropriate current from the inverter. The transient
response of the current loops determines the tracking
accuracy between the demanded and actual current
drawn from the inverter being tested. The passive components in the load emulation (i.e. the line inductors) act
to slow the response of the system.\par
The future course work will be focused on an industrial
induction motor drive, and to simulate virtual machine. It will be at real levels of voltage and current, the behavior of the actual machine. The results of testing a standard `off-the-shelf inverter with the virtual machine will be compared with those
obtained by testing the same inverter with the actual
machine.
